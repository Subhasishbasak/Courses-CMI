\documentclass[12pt,a4paper,oneside]{amsart}
%Title details
\usepackage{hyperref, fancyhdr}
\oddsidemargin -0.5in
\topmargin -0.3in
%\evensidemargin -0.3in
\textwidth 7.2in
\textheight 10in
\parindent=0.5in
\parskip 1.2mm
\pagestyle{fancy}
\title{Statement of Purpose}
\author{\small{Arnab Kar}}
\begin{document}

\maketitle
\fancyhead{}
\fancyfoot{}

\rhead{\small{Arnab Kar, Applying for Ph.D. in Physics, Fall 2009}}
\cfoot{\thepage}

I am applying for the admission to the Ph.D. programme in Physics. In general, my interests in Physics span various fields, but I am primarily interested in High Energy Physics and Quantum Optics. I desire to make significant contributions to my field, and also achieve excellence in teaching.

The techniques of logical deductions, approximations and assumptions used in physics to explain various phenomena have always delighted me. During my high school days I used to be a regular reader of the magazine \textit{Physics for you} and solved challenging problems from there. After my High School I enrolled in the B.Sc.(Hons) programme at \textit{Chennai Mathematical Institute (CMI)}. CMI's National Undergraduate Programme offers an academic and research oriented environment to students, who are taught by active researchers from the \textit{Institute of Mathematical Sciences (IMSc), Indira Gandhi Center for Atomic Research (IGCAR)} and other premier institutes of India in addition to CMI itself.

I have benefited from the all rounded curriculum\footnote{Course Details: \url{http://www.cmi.ac.in/teaching/courses.php?prog=bscp}} at CMI. Courses in group theory, linear algebra, calculus, complex analysis along with mathematical physics have helped me develop a strong foundation in mathematics. In my first year, I did courses on Classical Mechanics, Classical Electromagnetism,  Thermodynamics and Statistical Mechanics. I did a great deal of problem solving in my fundamental courses and referred to texts like \textit{Herbert Goldstein's} Classical Mechanics, \textit{Percival's} Introduction to Dynamics, \textit{David Griffith's} Introduction to Electrodynamics, \textit{Callen's} Thermodynamics and an Introduction to Thermostatistics and \textit{R.K. Pathria's} Statistical Mechanics. I have also done a course in General Relativity in the last semester where I read the \textit{Hartle's} book on Introduction to Einstein's General Relativity. 

In my third semester, I did my first course in Quantum Mechanics. I studied Non Relativistic Quantum Mechanics extensively from \textit{David Griffith's} Introduction to Quantum Mechanics and \textit{L.I. Schiff's} Quantum Mechanics. To have a better understanding of the mathematics used in the Quantum Mechanics I read the book of \textit{Dirac's} Principles of Quantum Mechanics. At the same time I was lucky to attend a series of lectures on Continuous Groups for physicists given by Prof. Mukunda in IMSc. During this course I learnt about Representation theory of Lie Groups. I read about the various space time groups and their representations namely the Galilean group, Lorentz group and Poincare group. At a later point of time I was able to appreciate those topics much better when I did a course on Relativistic Quantum Mechanics.
To know more about the foundations of quantum mechanics, I have independently read the first six chapters of Lectures on Quantum Theory: Mathematical and Structural Foundations by \textit{C.J.Isham} and intend to read more from it.

In the winter that followed that semester, I came to know about Non Hermitian operators. In this regard, I read the article by C.~M.~Bender and S.~Boettcher (Phys.~Rev.~Lett. 80, 5243 (1998)). They had replaced the condition of self adjointness which is necessary to have real and bounded eigen values of an operator by making the operator PT symmetric. I appreciated the importance of N while choosing the potential and how it determined whether the eigen values of the Hamiltonian will be real or complex. To find some application of this kind of a Hamiltonian, I read an article on the arXiv (quant-ph/0701141v1) by S. G. Rajeev on Dissipative Mechanics using Complex Valued Hamiltonians. The effect of damping for a simple harmonic oscillator was introduced by replacing natural frequency of oscillation by a complex number. It was shown that the only stable state of the system is the ground state having real eigen values. 

In my Relativistic Quantum Mechanics course, I read a few chapters of \textit{Bjorken and Drell's} book on Relativistic Quantum Mechanics and \textit{Sakurai's} Advanced Quantum Mechanics. I will be reading more on Quantum Field Theory, Particle Physics in the courses on these subjects in the current semester. I have already started reading \textit{Schweber's} Introduction to Relativistic Quantum Field theory. In the Atomic and Molecular Physics, I learnt how to calculate the lifetime of electrons undergoing spontaneous emission using creation and annihilation operators associated with the radiation field (an assembly of quantum harmonic oscillators). I also came to know about various ways to model many body quantum systems like the Hartree Fock model and the Thomas Fermi Statistical model. I was then inclined to do a  project\footnote{\url{http://www.cmi.ac.in/~arnabkar/project/thomas_fermi.pdf}} to verify the solutions to Thomas Fermi equation obtained by Bush and Cadwell (Phys.Rev. 38, 1898 (1931)) numerically. I solved the problem in GNU Octave using finite difference approximation to the second derivative. 

In the summer of 2008, having learnt about Scattering theory during my courses I did an experimental project\footnote{\url{http://www.cmi.ac.in/~arnabkar/project/irradiation.pdf}} in the Metal Physics Section of IGCAR, Kalpakkam. In this project I studied the difference between a pure Si crystal and a Si crystal implanted with hydrogen ions to a certain depth using Coincidence Doppler Broadening methods. The hydrogen ion implantation was carried out using a variable energy accelerator. The results showed that the chemical surrounding the annihilation site is more sensitive and affected. Since most positrons impeded and created open-volume defects in the valence region, as compared with that of the core electrons, the S parameter (sensitive to low momentum valence electrons) of irradiated sample increased significantly.

I have studied the Adiabatic Theorem in both Classical and Quantum Mechanics in detail. I then tried to find an analogy between them. I took the example of a simple harmonic oscillator and found that the ratio of the energy of the system to the frequency of oscillation turns out to be the adiabatic invariant (action variable for a classical system and quantum number (n) for a quantum system). I studied Berry's Phase and also read about the Aharonov-Bohm effect. I have read Aharonov and Anandan's paper which removed the condition of adiabaticity from Berry's phase. However Pancharatnam's work on interference of polarized light demonstrated that even a non unitary evolution would lead to the addition of a phase factor.

Amongst the research going on in ROchester University, the research on foundations of quantum field theory in High Energy Physics department,interests me a lot. After reading a few articles on the new field of complex quantum mechanics, I am eager to know more about it and it's applications of this formalism to information theory. The use of a simple system such as a harmonic oscillator to show how interaction with the environment leads to energy loss which cannot be neglected amazed me. I would like to work on the foundations of quantum mechanics, study the importance of degree of entanglement in a quantum system which are all being studied in great details in the Quantum Optics department. With my interests centered around Quantum Physics and its applications to various areas, University of Rochester is the best choice for my further studies.

It is said that teaching is an integrated part of research. I have always enjoyed teaching others and presented\footnote{\url{http://www.cmi.ac.in/\~arnabkar/talk/talk.html}} few topics on various occasions. During my General Relativity course I presented the topic of Rotating Black holes to my class. As a part of student talks held in my college I spoke about Special Relativity drawing reference to Michelson Morley experiment. On one occasion I spoke about the Adiabatic theorem in Classical and Quantum Mechanics. Based on these experiences, and after having been appreciated by the audience as a good speaker I have gained confidence on my abilities as a teaching assistant.

Keeping in mind my long term objective, I would like to pursue a Ph.D. as it would enable me to delve deeper into the subject I am interested in. I believe, University of Rochester with its research facilities and illustrious faculty is the best choice for me. With my diverse academic background and zeal for research, I am sure that, given an opportunity to work with the preeminent researchers of the University, I will contribute substantially to my field of work.

\end{document}
